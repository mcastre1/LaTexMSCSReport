\chapter{MSCS Project Proposal} %\label{Atmosniffer Overview}
%\vspace{-7mm}
%\bigskip
This chapter provides the purpose, approach, research, and criteria on the approved project proposal.
\section{Purpose}
The main purpose of this application was to make a more user friendly AtmoSniffer Device Interface that could be easily accessed by both, developers and users.
Moreover, the main audience for the application is the research team at WSU. For them to be able to visualize data sent by a device, in a graph format, and to communicate and control the device for calibration purposes.

Other than building an application that people can easily use, I wanted to learn python, a new framework, such as qt, and how to professionally document a project.

\section{Approach}
To develop the application, I went through each step of the Software Development Life Cycle. With the help and permission of Dr. Valle, I created a GitHub repository to hold a copy of the old desktop application and start building from there. At first, I tried continuing and building up from the old desktop application, but I rather quickly found out that the old project lacked proper documentation and it was a much better idea to start from scratch rather than spend a whole month trying to decyphering it. I still kept the old project there under a legacy folder as a reference.

I was able to create a GUI and test it only with the help of python libraries. I met with users throughout and scheduled sprints to develop features.
\section{Research}
Research was focused on learning a QT, a new framework, and pythong. I used pyqt5 library which is the binding between qt and python.

The application needed to be able to handle a lot of data quickly. Python came with a big number of libraries used for data visualization, analysis, and parsing. This is the reason we chose it, but also because I have always wanted to learn it.

One more reason I found to do this project was for me to learn and experience a professional approach to software development. Including clean code, documentation, life cycle, testing, and a lot of messing up and debugging. 

The optimal goal of this project is to visualize a large amount of data.

\section{Criteria}
Application goals and milestones
\begin{itemize}
	\item Read data from SD card
	\item Visualize data from SD files
	\item Suppport bluetooth and radio connectivity.
		\begin{itemize}
		\item Connec to device
		\item Receive data from a device through bluetooth and/or radio signals
		\end{itemize}
	\item Visualize data from broadcast records.
		\begin{itemize}
		\item Create and close graphs
		\item Give the user the option to choose which type of graph they would like to see.
		\end{itemize}	
	\item Send commands to the connected device through bluetooth and/or radio signals
		\begin{itemize}
		\item Turn ON/OFF device components
		\item Enable extra debugging
		\item Reset sensors
		\end{itemize}
	\item Documentation
		\begin{itemize}
		\item Full documentation of code, following best standards.
		\item Documentation on how to use the application
		\end{itemize}
	\item Unit Testing
		\begin{itemize}
		\item This includes code and gui components of the application
		\end{itemize}
	\item Support Multiple OS
		\begin{itemize}
		\item Create executables/installers for various OS
		\end{itemize}
\end{itemize}
